%!TEX program = xelatex
% ─────────────────────────────────────────────────────────
% MOEA/D 科研知识多目标权衡关系分析报告
% 编译命令: xelatex tradeoff_report.tex
% ─────────────────────────────────────────────────────────
\documentclass[11pt, a4paper]{article}

% ── 页面 ──────────────────────────────────────────────────
\usepackage[a4paper,
            top=22mm, bottom=20mm,
            left=18mm, right=18mm,
            headheight=15pt]{geometry}

% ── 中文支持 ──────────────────────────────────────────────
\usepackage{fontspec}
\usepackage{xeCJK}
\setCJKmainfont{PingFang SC}
\setCJKmonofont{STKaiti}
\setmainfont{Georgia}
\setsansfont{Helvetica Neue}

% ── 数学 ──────────────────────────────────────────────────
\usepackage{amsmath, amssymb}

% ── 颜色 ──────────────────────────────────────────────────
\usepackage[dvipsnames, table, x11names]{xcolor}
\definecolor{CBlue}  {HTML}{1565C0}
\definecolor{CTeal}  {HTML}{00695C}
\definecolor{CAmber} {HTML}{E65100}
\definecolor{CPurple}{HTML}{6A1B9A}
\definecolor{CRed}   {HTML}{C62828}
\definecolor{CGreen} {HTML}{2E7D32}
\definecolor{CGray}  {HTML}{546E7A}
\definecolor{CLGray} {HTML}{ECEFF1}
\definecolor{CDark}  {HTML}{212121}
\definecolor{CPink}  {HTML}{AD1457}
\definecolor{CBg}    {HTML}{E3F2FD}
\definecolor{CBg2}   {HTML}{E8F5E9}
\definecolor{CAmberL}{HTML}{FFF8E1}
\definecolor{CRedL}  {HTML}{FCE4EC}
\definecolor{CGreenL}{HTML}{E8F5E9}
\definecolor{CPurpL} {HTML}{F3E5F5}
\definecolor{CBlueL} {HTML}{E3F2FD}
\definecolor{CBlueHdr}{HTML}{1A237E}

% ── 图形 ──────────────────────────────────────────────────
\usepackage{tikz}
\usetikzlibrary{positioning, arrows.meta, calc, shapes.geometric,
                decorations.pathreplacing, backgrounds, fit, matrix}
\usepackage{pgfplots}
\pgfplotsset{compat=1.18}
\usepackage{pgfplotstable}

% ── 表格 ──────────────────────────────────────────────────
\usepackage{booktabs}
\usepackage{tabularx}
\usepackage{multirow}
\usepackage{makecell}
\usepackage{array}
\usepackage{colortbl}
\usepackage{longtable}

% ── 排版 ──────────────────────────────────────────────────
\usepackage{fancyhdr}
\usepackage{titlesec}
\usepackage{enumitem}
\usepackage{caption}
\usepackage{subcaption}
\usepackage{float}
\usepackage{microtype}
\usepackage{parskip}
\usepackage{hyperref}

\hypersetup{
  colorlinks=true, linkcolor=CBlue, urlcolor=CBlue,
  pdftitle={MOEA/D 科研知识多目标权衡分析报告},
  pdfauthor={MOEA/D × Claude 系统}
}

% ── 页眉页脚 ─────────────────────────────────────────────
\pagestyle{fancy}
\fancyhf{}
\fancyhead[L]{\small\color{CGray} MOEA/D 科研知识多目标权衡分析报告}
\fancyhead[R]{\small\color{CGray} 第 \thepage\ 页}
\fancyfoot[C]{\small\color{CGray} MOEA/D $\times$ Claude $\cdot$ 跨学科长尾科研知识演化系统}
\renewcommand{\headrulewidth}{0.3pt}
\renewcommand{\footrulewidth}{0pt}

% ── 标题样式 ──────────────────────────────────────────────
\titleformat{\section}[block]
  {\normalfont\large\bfseries\color{white}}
  {}
  {0pt}
  {\colorbox{CBlue}{\parbox{\dimexpr\linewidth-2\fboxsep\relax}{#1}}}
\titlespacing*{\section}{0pt}{12pt}{6pt}

\titleformat{\subsection}[hang]
  {\normalfont\normalsize\bfseries\color{CTeal}}
  {\thesubsection}{1em}{}
\titlespacing*{\subsection}{0pt}{8pt}{3pt}

% ── 表格列格式 ───────────────────────────────────────────
\newcolumntype{C}[1]{>{\centering\arraybackslash}p{#1}}
\newcolumntype{L}[1]{>{\raggedright\arraybackslash}p{#1}}
\newcolumntype{R}[1]{>{\raggedleft\arraybackslash}p{#1}}

% ── 节标题宏 ─────────────────────────────────────────────
\newcommand{\secnum}[2]{%
  \section*{\makebox[\linewidth][l]{%
    \colorbox{CBlue}{\parbox{\dimexpr\linewidth-2\fboxsep\relax}{%
      \hspace{4pt}\textcolor{white}{\textbf{#1\quad #2}}\hspace{4pt}}}}}%
  \vspace{-4pt}%
}

% ─────────────────────────────────────────────────────────
\begin{document}
% ─────────────────────────────────────────────────────────

% ══════════════════════════════════
% 封面
% ══════════════════════════════════
\thispagestyle{fancy}
\vspace*{8mm}
\begin{center}
  {\LARGE\bfseries\color{CBlue} 科研长尾知识多目标权衡关系}\\[4pt]
  {\Large\bfseries\color{CBlue} 深度分析报告}\\[8pt]
  {\normalsize\color{CGray}
    基于 MOEA/D $\times$ Claude 演化系统 $\cdot$ v2(7目标版)$+$ v3(可行性强化版)}\\[2pt]
  {\normalsize\color{CGray}
    样本量:65 个多样化长尾研究知识 $\cdot$ Pareto 解集:26 个}
\end{center}
\vspace{4pt}
\noindent\textcolor{CLGray}{\rule{\linewidth}{1pt}}
\vspace{4pt}

% 摘要表
\begin{table}[H]
\centering
\small
\setlength{\tabcolsep}{6pt}
\renewcommand{\arraystretch}{1.3}
\begin{tabularx}{\linewidth}{>{\bfseries\color{CDark}}L{3.8cm}
                              C{1.6cm} C{1.6cm}
                              >{\raggedright\arraybackslash\color{CDark}}X}
  \rowcolor{CBlueHdr}
  \textcolor{white}{\textbf{核心发现}} &
  \textcolor{white}{\textbf{$r$ 值}} &
  \textcolor{white}{\textbf{类型}} &
  \textcolor{white}{\textbf{含义}} \\
  \rowcolor{CGreenL}
  可行性 $\leftrightarrow$ 合理性 & \textcolor{CGreen}{$+0.685$} & 强协同 &
    验证轴:理论严密 $\Longleftrightarrow$ 方法可行 \\
  \rowcolor{CRedL}
  长尾度 $\leftrightarrow$ 可行性 & \textcolor{CRed}{$-0.502$} & 强对抗 &
    最核心权衡:越小众越难落地 \\
  \rowcolor{CRedL}
  合理性 $\leftrightarrow$ 长尾度 & \textcolor{CRed}{$-0.489$} & 对抗 &
    主流框架严密,但往往不够长尾 \\
  \rowcolor{CAmberL}
  前沿性 $\leftrightarrow$ 可行性 & \textcolor{CAmber}{$-0.447$} & 对抗 &
    突破性想法领先于实验能力 \\
  \rowcolor{CGreenL}
  长尾度 $\leftrightarrow$ 前沿性 & \textcolor{CTeal}{$+0.325$} & 弱协同 &
    探索轴:小众方向天然趋向前沿 \\
  \rowcolor{CPurpL}
  前沿性 $\leftrightarrow$ 合理性 & \textcolor{CPurple}{$-0.088$} & \textbf{独立} &
    \textbf{★ 最反直觉:前沿不必牺牲严谨} \\
\end{tabularx}
\caption*{表~0\quad 核心发现摘要}
\end{table}

\vspace{2pt}
\noindent\textcolor{CLGray}{\rule{\linewidth}{0.5pt}}

% ══════════════════════════════════
\secnum{01}{四目标结构:探索轴 vs 验证轴}
% ══════════════════════════════════

实验数据揭示出四个核心目标形成清晰的\textbf{二维拉力场}:
\textcolor{CGreen}{长尾度}与\textcolor{CAmber}{前沿性}构成「探索轴」(正相关,$r=+0.33$);
\textcolor{CRed}{可行性}与\textcolor{CPink}{合理性}构成「验证轴」(强正相关,$r=+0.69$)。
两轴之间存在系统性对抗关系。

\begin{figure}[H]
\centering
\begin{tikzpicture}[
  node/.style={circle, minimum size=2.0cm, align=center,
               font=\small\bfseries, text=white, inner sep=3pt},
  synergy/.style={line width=2pt},
  tension/.style={line width=1.8pt, dashed},
  indep/.style={line width=1pt, dotted, CGray},
]
% 四节点
\node[node, fill=CGreen!85!black]  (LT) at (-3.5, 1.2) {长尾度};
\node[node, fill=CAmber]            (FR) at ( 3.5, 1.2) {前沿性};
\node[node, fill=CRed]              (FE) at (-3.5,-1.2) {可行性};
\node[node, fill=CPink]             (RI) at ( 3.5,-1.2) {合理性};

% 探索轴(协同)
\draw[synergy, CTeal]
  (LT) -- (FR)
  node[midway, above, font=\small\color{CTeal}] {$r=+0.33$}
  node[midway, below, font=\scriptsize\color{CTeal}] {协同·探索轴};
% 验证轴(强协同)
\draw[synergy, CBlue, line width=2.5pt]
  (FE) -- (RI)
  node[midway, below, font=\small\color{CBlue}] {$r=+0.69$}
  node[midway, above, font=\scriptsize\color{CBlue}] {强协同·验证轴};
% 主对抗轴
\draw[tension, CRed]
  (LT) -- (FE)
  node[midway, left, font=\small\color{CRed}] {$r=-0.50$};
\draw[tension, CAmber]
  (FR) -- (FE)
  node[midway, right=2pt, font=\small\color{CAmber}] {$r=-0.45$};
\draw[tension, CPurple]
  (LT) to[bend left=15] (RI)
  node[midway, above=6pt, font=\small\color{CPurple}] {$r=-0.49$};
% 独立轴
\draw[indep]
  (FR) -- (RI)
  node[midway, right, font=\small\color{CGray}] {$r=-0.09$(独立)};

% 图例
\begin{scope}[shift={(-4.8,-3.2)}]
  \draw[CTeal, line width=2pt] (0,0) -- (1,0);
  \node[right, font=\scriptsize, CDark] at (1.1,0) {协同};
  \draw[CRed, line width=1.8pt, dashed] (2.8,0) -- (3.8,0);
  \node[right, font=\scriptsize, CDark] at (3.9,0) {对抗};
  \draw[CGray, dotted] (5.6,0) -- (6.6,0);
  \node[right, font=\scriptsize, CDark] at (6.7,0) {独立};
\end{scope}
\end{tikzpicture}
\caption{四目标结构关系示意图(线型/颜色代表关系类型)}
\end{figure}

% ══════════════════════════════════
\secnum{02}{全目标相关系数热图(65个样本)}
% ══════════════════════════════════

热图展示7个目标两两之间的 Pearson 相关系数。深蓝=强正相关(协同),深红=强负相关(权衡),白色$\approx$独立。

\begin{figure}[H]
\centering
\begin{tikzpicture}
\begin{axis}[
  width=11cm, height=11cm,
  enlargelimits=false,
  colormap={bluered}{
    rgb255(0)=(198,40,40);
    rgb255(45)=(255,255,255);
    rgb255(100)=(21,101,192)
  },
  colorbar,
  colorbar style={
    ylabel={Pearson $r$},
    ytick={-1,-0.5,0,0.5,1},
  },
  point meta min=-1, point meta max=1,
  xtick={0,1,2,3,4,5,6},
  ytick={0,1,2,3,4,5,6},
  xticklabels={知识,社会,长尾,跨学,前沿,可行,合理},
  yticklabels={知识,社会,长尾,跨学,前沿,可行,合理},
  x tick label style={rotate=30, anchor=east, font=\small},
  y tick label style={font=\small},
  tick align=inside,
  xlabel style={font=\small},
  ylabel style={font=\small},
]
% 热图数据(行从 y=0→6 = 知识→合理)
\addplot[matrix plot*, mesh/cols=7, point meta=explicit] table[meta=C] {
  x y C
  0 6  1.000
  1 6  0.132
  2 6 -0.125
  3 6  0.015
  4 6  0.553
  5 6  0.105
  6 6  0.580
  0 5  0.132
  1 5  1.000
  2 5 -0.451
  3 5 -0.045
  4 5  0.051
  5 5 -0.077
  6 5  0.151
  0 4 -0.125
  1 4 -0.451
  2 4  1.000
  3 4  0.451
  4 4  0.325
  5 4 -0.502
  6 4 -0.489
  0 3  0.015
  1 3 -0.045
  2 3  0.451
  3 3  1.000
  4 3  0.371
  5 3 -0.367
  6 3 -0.266
  0 2  0.553
  1 2  0.051
  2 2  0.325
  3 2  0.371
  4 2  1.000
  5 2 -0.447
  6 2 -0.088
  0 1  0.105
  1 1 -0.077
  2 1 -0.502
  3 1 -0.367
  4 1 -0.447
  5 1  1.000
  6 1  0.685
  0 0  0.580
  1 0  0.151
  2 0 -0.489
  3 0 -0.266
  4 0 -0.088
  5 0  0.685
  6 0  1.000
};

% 标注数值(只标注 |r| > 0.3 的)
\pgfplotsset{every node near coord/.append style={font=\tiny}}
\node[font=\tiny, white] at (axis cs:4,6) {$+0.55$};
\node[font=\tiny, white] at (axis cs:6,6) {$+0.58$};
\node[font=\tiny, white] at (axis cs:1,5) {$-0.45$};
\node[font=\tiny, white] at (axis cs:2,4) {$+0.45$};
\node[font=\tiny, white] at (axis cs:3,4) {$+0.33$};
\node[font=\tiny, white] at (axis cs:5,4) {$-0.50$};
\node[font=\tiny, white] at (axis cs:6,4) {$-0.49$};
\node[font=\tiny       ] at (axis cs:2,3) {$+0.45$};
\node[font=\tiny       ] at (axis cs:3,2) {$+0.37$};
\node[font=\tiny       ] at (axis cs:5,2) {$-0.45$};
\node[font=\tiny, white] at (axis cs:6,1) {$+0.69$};
\node[font=\tiny, white] at (axis cs:5,1) {$-0.50$};
\end{axis}
\end{tikzpicture}
\caption{7目标两两 Pearson 相关系数热图(深蓝=协同,深红=对抗)}
\end{figure}

% ══════════════════════════════════
\secnum{03}{四象限分布:长尾度 $\times$ 可行性}
% ══════════════════════════════════

以\textbf{长尾度}(横轴)和\textbf{可行性}(纵轴)定义四象限。
每个点的\textbf{颜色}代表合理性(红$\to$绿),\textbf{圆圈大小}代表前沿性。

\begin{figure}[H]
\centering
\begin{tikzpicture}
\begin{axis}[
  width=12cm, height=8.5cm,
  xmin=5.5, xmax=10.5, ymin=-0.5, ymax=10.5,
  xlabel={长尾度},
  ylabel={可行性},
  xlabel style={font=\small},
  ylabel style={font=\small},
  tick label style={font=\small},
  grid=none,
  axis line style={CGray},
]
% 象限背景
\fill[CAmberL, opacity=0.7]
  (axis cs:5.5,5) rectangle (axis cs:8,10.5);
\fill[CRedL, opacity=0.7]
  (axis cs:8,5)   rectangle (axis cs:10.5,10.5);
\fill[CGreenL, opacity=0.7]
  (axis cs:5.5,-0.5) rectangle (axis cs:8,5);
\fill[CPurpL, opacity=0.7]
  (axis cs:8,-0.5)   rectangle (axis cs:10.5,5);

% 象限标签
\node[font=\scriptsize\color{CGray!70!black}, align=center]
  at (axis cs:6.7,7.5) {低长尾+高可行\\成熟研究型};
\node[font=\scriptsize\color{CGray!70!black}, align=center]
  at (axis cs:9.3,7.5) {高长尾+高可行\\★黄金三角型};
\node[font=\scriptsize\color{CGray!70!black}, align=center]
  at (axis cs:6.7,2.0) {低长尾+低可行\\理论假说型};
\node[font=\scriptsize\color{CGray!70!black}, align=center]
  at (axis cs:9.3,2.0) {高长尾+低可行\\天马行空型};

% 中位线
\draw[CGray, dashed, thin] (axis cs:8,-0.5) -- (axis cs:8,10.5);
\draw[CGray, dashed, thin] (axis cs:5.5,5)  -- (axis cs:10.5,5);

% 散点(颜色=合理性 r/10,大小=前沿性)
% 格式: \addplot[scatter, ...] coordinates {...};
% 用不同颜色组模拟合理性渐变(低=红,中=橙,高=绿)
% 合理性 3-4 (低)
\addplot[only marks, mark=*, mark size=4pt,
         mark options={fill=CRed!80!white, draw=white, line width=0.5pt}]
  coordinates {
    (10,1) (10,2) (10,2) (8,6) (8,6) (9,3) (10,3) (9,3)
  };
% 合理性 5-6 (中低)
\addplot[only marks, mark=*, mark size=5pt,
         mark options={fill=CAmber!80!white, draw=white, line width=0.5pt}]
  coordinates {
    (9,5) (9,4) (9,4) (9,6) (8,5) (9,4) (7,4) (8,6) (8,6)
    (9,5) (9,5) (10,5) (9,5)
  };
% 合理性 6-7 (中高)
\addplot[only marks, mark=*, mark size=6pt,
         mark options={fill=CTeal!70!white, draw=white, line width=0.5pt}]
  coordinates {
    (9,9) (9,9) (9,7) (10,8) (9,6) (9,6) (9,8) (9,8)
    (8,8) (9,6) (8,6) (8,7) (8,5) (9,6) (8,7) (8,7)
    (9,6) (9,7) (9,7) (10,4) (9,3)
  };
% 合理性 7-8 (高)
\addplot[only marks, mark=*, mark size=7pt,
         mark options={fill=CGreen!80!black, draw=white, line width=0.5pt}]
  coordinates {
    (8,8) (8,9) (8,8) (9,8) (7,9) (9,8) (8,5)
    (7,7) (7,9) (8,8) (9,6) (9,3) (8,8) (8,8)
    (8,7) (7,8) (8,7) (8,8) (9,6) (8,7)
  };

\end{axis}
\end{tikzpicture}

\smallskip
\noindent\hfill
\begin{tikzpicture}[baseline=-3pt]
  \fill[CRed!80!white]  (0,0) circle (4pt); \node[right, font=\scriptsize] at (5pt,0) {合理=3--4(低)};
  \fill[CAmber!80!white](2cm,0) circle (5pt); \node[right, font=\scriptsize] at (2cm+6pt,0) {合理=5--6(中)};
  \fill[CTeal!70!white] (4.2cm,0) circle (6pt); \node[right, font=\scriptsize] at (4.2cm+7pt,0) {合理=6--7(中高)};
  \fill[CGreen!80!black](6.6cm,0) circle (7pt); \node[right, font=\scriptsize] at (6.6cm+8pt,0) {合理=7--8(高)};
\end{tikzpicture}\hfill\null

\caption{四象限散点图(颜色=合理性,圆圈大小=前沿性)}
\end{figure}

\begin{table}[H]
\centering\small
\setlength{\tabcolsep}{5pt}
\renewcommand{\arraystretch}{1.25}
\begin{tabularx}{\linewidth}{L{2.8cm} C{2.2cm} C{1.2cm} C{1.8cm} C{1.8cm} X}
  \rowcolor{CDark}
  \textcolor{white}{\textbf{象限}} &
  \textcolor{white}{\textbf{特征}} &
  \textcolor{white}{\textbf{$n$}} &
  \textcolor{white}{\textbf{前沿性均值}} &
  \textcolor{white}{\textbf{合理性均值}} &
  \textcolor{white}{\textbf{代表类型}} \\
  \rowcolor{CAmberL}
  高长尾+高可行 & 长尾$\geq$8.5, 可行$\geq$5 & 17 & 6.9 & 6.6 &
    \textcolor{CAmber}{\textbf{★ 黄金三角型}} \\
  \rowcolor{CRedL}
  高长尾+低可行 & 长尾$\geq$8.5, 可行$<$5  & 21 & 7.8 & 5.3 & 天马行空型 \\
  \rowcolor{CGreenL}
  低长尾+高可行 & 长尾$<$8.5, 可行$\geq$5  & 21 & 6.8 & 7.0 & 成熟研究型 \\
  \rowcolor{CPurpL}
  低长尾+低可行 & 长尾$<$8.5, 可行$<$5     & 6  & 7.2 & 6.0 & 理论假说型 \\
\end{tabularx}
\caption{四象限统计(以长尾度/可行性中位数划分)}
\end{table}

% ══════════════════════════════════
\secnum{04}{最反直觉发现:前沿性 $\leftrightarrow$ 合理性 独立($r=-0.09$)}
% ══════════════════════════════════

\begin{minipage}[t]{0.48\linewidth}
\vspace{0pt}
\textbf{前沿性与合理性几乎不相关}($r=-0.088$),与「大胆创新必然不严谨」的直觉相反。

真正制约两者共存的是\textbf{可行性瓶颈}:高前沿$+$高合理的研究虽然概念上自洽,但当前技术往往无法同时撑起「新颖」$+$「可验证」。

65个样本中只有\textbf{1个「理论圣杯型」}个体:

\begin{center}
\colorbox{CBlueL}{\parbox{0.85\linewidth}{\centering\small
  \textcolor{CBlue}{\textbf{古DNA表观遗传印记重建}}\\
  古遗传学$\times$表观基因组学\\[4pt]
  \textcolor{CGreen}{长尾=9} \quad \textcolor{CAmber}{前沿=9} \quad
  \textcolor{CRed}{可行=6} \quad \textcolor{CPink}{合理=8}
}}
\end{center}

\vspace{4pt}
这说明同时高前沿$+$高合理$+$高长尾是 \textbf{Pareto 极端点},极其罕见但确实存在。
\end{minipage}%
\hfill
\begin{minipage}[t]{0.48\linewidth}
\vspace{0pt}
% 雷达图:三类典型画像
\begin{tikzpicture}
\begin{polaraxis}[
  width=6.5cm, height=6.5cm,
  xtick={0,90,180,270},
  xticklabels={长尾度, 前沿性, 可行性, 合理性},
  x tick label style={font=\small},
  ymin=0, ymax=10,
  ytick={2,4,6,8,10},
  yticklabels={2,4,6,8,10},
  y tick label style={font=\scriptsize, color=CGray},
  grid=both,
  grid style={thin, CLGray},
  axis line style={CGray},
]
% A类: 天马行空型
\addplot[CRed, line width=1.5pt, mark=*, mark size=2pt,
         mark options={fill=CRed}]
  coordinates {(0,9.5)(90,8)(180,2)(270,4.5)(360,9.5)};
% B类: 黄金三角型
\addplot[CAmber!80!black, line width=1.5pt, mark=square*, mark size=2pt,
         mark options={fill=CAmber}]
  coordinates {(0,8)(90,6.5)(180,8.5)(270,7.5)(360,8)};
% C类: 理论圣杯型
\addplot[CBlue, line width=2pt, mark=diamond*, mark size=2.5pt,
         mark options={fill=CBlue}]
  coordinates {(0,9)(90,9)(180,6)(270,8)(360,9)};
\end{polaraxis}
\end{tikzpicture}

\vspace{2pt}
{\scriptsize
\textcolor{CRed}{$\bullet$} A类·天马行空型\quad
\textcolor{CAmber}{$\blacksquare$} B类·黄金三角型\quad
\textcolor{CBlue}{$\blacklozenge$} C类·理论圣杯型}
\end{minipage}

\captionof{figure}{三类典型研究画像雷达对比(四维:长尾/前沿/可行/合理)}

\vspace{6pt}

% ══════════════════════════════════
\secnum{05}{v3 演化趋势(6代·四目标均值)}
% ══════════════════════════════════

强化约束算法(v3)运行6代的四目标均值演化曲线。
实线=强化目标(可行性/合理性),虚线=被动调整目标(长尾度/前沿性)。

\begin{figure}[H]
\centering
\begin{tikzpicture}
\begin{axis}[
  width=13cm, height=6cm,
  xmin=-0.2, xmax=6.2,
  ymin=4.5, ymax=9.5,
  xlabel={演化代数},
  ylabel={均值(分)},
  xlabel style={font=\small},
  ylabel style={font=\small},
  tick label style={font=\small},
  xtick={0,1,2,3,4,5,6},
  xticklabels={第0代,第1代,第2代,第3代,第4代,第5代,第6代},
  x tick label style={rotate=20, anchor=east, font=\scriptsize},
  ytick={5,6,7,8,9},
  legend pos=outer north east,
  legend style={font=\small, row sep=1pt},
  grid=major,
  grid style={thin, CLGray},
]
% 长尾度
\addplot[CGreen!80!black, dashed, line width=1.5pt, mark=o, mark size=3pt,
         mark options={fill=CGreen!80!black}]
  coordinates {(0,8.45)(1,8.30)(2,8.35)(3,8.15)(4,8.35)(5,8.30)(6,8.30)};
\addlegendentry{长尾度}
% 前沿性
\addplot[CAmber, dashed, line width=1.5pt, mark=square, mark size=3pt,
         mark options={fill=CAmber}]
  coordinates {(0,6.60)(1,6.60)(2,6.90)(3,7.15)(4,7.35)(5,7.35)(6,7.50)};
\addlegendentry{前沿性}
% 可行性
\addplot[CRed, line width=2pt, mark=*, mark size=3pt,
         mark options={fill=CRed}]
  coordinates {(0,7.65)(1,7.30)(2,7.85)(3,7.55)(4,7.40)(5,7.40)(6,7.35)};
\addlegendentry{可行性(强化)}
% 合理性
\addplot[CPink, line width=2pt, mark=diamond*, mark size=3pt,
         mark options={fill=CPink}]
  coordinates {(0,7.25)(1,6.65)(2,6.75)(3,6.75)(4,6.90)(5,6.85)(6,7.00)};
\addlegendentry{合理性(强化)}
\end{axis}
\end{tikzpicture}
\caption{v3 四目标均值演化趋势(第0代=初始种群)}
\end{figure}

% ══════════════════════════════════
\secnum{06}{算法改进代价分析:v2 vs v3}
% ══════════════════════════════════

通过强制提升可行性$+$合理性(v3),演化方向从「宇宙量子假说」漂移到「冰川微生物基因组、古DNA表观遗传」等可落地方向。代价是长尾度和前沿性各降约0.7分。

\begin{figure}[H]
\centering
\begin{tikzpicture}
\begin{axis}[
  ybar=3pt,
  width=11cm, height=6cm,
  bar width=14pt,
  ymin=2, ymax=11,
  ytick={4,6,8,10},
  ylabel={均值(分)},
  ylabel style={font=\small},
  symbolic x coords={长尾度, 前沿性, 可行性, 合理性},
  xtick=data,
  x tick label style={font=\normalsize},
  tick label style={font=\small},
  legend style={at={(0.98,0.98)}, anchor=north east, font=\small},
  grid=major,
  grid style={thin, CLGray},
  nodes near coords,
  nodes near coords style={font=\scriptsize, anchor=south},
  every node near coord/.append style={yshift=2pt},
]
% v2
\addplot[fill=CBlue!25!white, draw=CBlue!60!white, line width=0.5pt]
  coordinates {(长尾度,9.1)(前沿性,8.2)(可行性,4.7)(合理性,5.9)};
\addlegendentry{v2(无约束)}
% v3
\addplot[fill=CTeal, draw=CTeal!80!black, line width=0.5pt]
  coordinates {(长尾度,8.3)(前沿性,7.5)(可行性,7.4)(合理性,7.0)};
\addlegendentry{v3(强化约束)}
\end{axis}
\end{tikzpicture}
\caption{v2 vs v3 四目标均值对比柱状图(深色=v3,浅色=v2)}
\end{figure}

\begin{table}[H]
\centering\small
\setlength{\tabcolsep}{5pt}
\renewcommand{\arraystretch}{1.3}
\begin{tabularx}{\linewidth}{L{2.4cm} C{1.8cm} C{1.8cm} C{1.6cm} X}
  \rowcolor{CDark}
  \textcolor{white}{\textbf{维度}} &
  \textcolor{white}{\textbf{v2 无约束}} &
  \textcolor{white}{\textbf{v3 强化}} &
  \textcolor{white}{\textbf{变化}} &
  \textcolor{white}{\textbf{解读}} \\
  \rowcolor{CGreenL}
  长尾度均值 & 9.1 & 8.3 & $-0.8$ & 略有下降,长尾性仍很高 \\
  \rowcolor{CGreenL}
  前沿性均值 & 8.2 & 7.5 & $-0.7$ & 轻微代价,仍属高前沿 \\
  \rowcolor{CBlueL}
  可行性均值 & 4.7 & 7.4 & \textcolor{CGreen}{\textbf{$+2.7$ ★}} & 大幅提升,核心改进目标 \\
  \rowcolor{CBlueL}
  合理性均值 & 5.9 & 7.0 & \textcolor{CGreen}{\textbf{$+1.1$ ★}} & 显著提升,理论基础更扎实 \\
  \rowcolor{CLGray}
  综合均值   & 7.1 & 7.4 & $+0.3$ & 整体质量提升 \\
  \rowcolor{CLGray}
  Pareto 解数 & 17 & 9   & $-8$   & 约束收紧,解集规模缩小 \\
  \rowcolor{CLGray}
  代表研究方向 & \multicolumn{2}{c}{暗物质拓扑 $\to$ 古DNA表观遗传}
              & --- & 从纯理论转向可实验验证 \\
\end{tabularx}
\caption{v2 与 v3 最终种群关键指标对比}
\end{table}

% ══════════════════════════════════
\secnum{07}{Pareto 最优解:各目标极值代表}
% ══════════════════════════════════

\begin{table}[H]
\centering\small
\setlength{\tabcolsep}{4pt}
\renewcommand{\arraystretch}{1.35}
\begin{tabularx}{\linewidth}{
  L{1.8cm} L{4.4cm} L{4.0cm}
  C{1cm} C{1cm} C{1cm} C{1cm} C{1cm} C{1cm}}
  \rowcolor{CDark}
  \textcolor{white}{\textbf{类型}} &
  \textcolor{white}{\textbf{研究主题}} &
  \textcolor{white}{\textbf{领域}} &
  \textcolor{white}{\textbf{长尾}} &
  \textcolor{white}{\textbf{前沿}} &
  \textcolor{white}{\textbf{可行}} &
  \textcolor{white}{\textbf{合理}} &
  \textcolor{white}{\textbf{知识}} &
  \textcolor{white}{\textbf{社会}} \\
  \rowcolor{CGreenL}
  \textcolor{CGreen}{\textbf{最高长尾}} &
  暗物质核反冲驱动的拓扑光子芯片纠错增强 &
  暗物质量子光子拓扑学 &
  10 & 10 & 2 & 4 & 7 & 6 \\
  \rowcolor{CAmberL}
  \textcolor{CAmber}{\textbf{最高前沿}} &
  暗物质核反冲驱动的拓扑光子芯片纠错增强 &
  暗物质量子光子拓扑学 &
  10 & 10 & 2 & 4 & 7 & 6 \\
  \rowcolor{CRedL}
  \textcolor{CRed}{\textbf{最高可行}} &
  工业废水中的定向进化实验 &
  环境微生物学--合成生物学 &
  7 & 7 & 9 & 8 & 8 & 6 \\
  \rowcolor{CBlueL}
  \textcolor{CPink}{\textbf{最高合理}} &
  古DNA中的表观遗传印记重建 &
  古遗传学--表观基因组学 &
  9 & 9 & 6 & 8 & 9 & 6 \\
\end{tabularx}
\caption{Pareto 前沿中各目标极值代表个体}
\end{table}

% ══════════════════════════════════
\secnum{08}{核心结论与科研选题启示}
% ══════════════════════════════════

\begin{table}[H]
\centering\small
\setlength{\tabcolsep}{5pt}
\renewcommand{\arraystretch}{1.5}
\begin{tabularx}{\linewidth}{C{0.5cm} L{2.2cm} X}
  \rowcolor{CDark}
  \textcolor{white}{\textbf{\#}} &
  \textcolor{white}{\textbf{维度}} &
  \textcolor{white}{\textbf{结论}} \\
  \rowcolor{CRedL}
  \textbf{\textcircled{\small 1}} &
  \textcolor{CRed}{\textbf{最核心权衡}} &
  长尾度 $\leftrightarrow$ 可行性($r=-0.50$)是整个系统中最难同时满足的对立关系。越小众的研究方向,越缺乏现成技术支撑——这是长尾知识的本质张力。\\
  \rowcolor{CBlueL}
  \textbf{\textcircled{\small 2}} &
  \textcolor{CBlue}{\textbf{验证轴统一}} &
  可行性与合理性高度协同($r=+0.69$),应作为整体「可研究性」评估,而非独立权衡。扎实的理论假设本身就需要具体实验设计来支撑。\\
  \rowcolor{CPurpL}
  \textbf{\textcircled{\small 3}} &
  \textcolor{CPurple}{\textbf{最反直觉}} &
  前沿性与合理性几乎独立($r=-0.09$)。制约「高前沿$+$高合理」共存的真正瓶颈是可行性不足,而非两者之间的内在矛盾。\\
  \rowcolor{CAmberL}
  \textbf{\textcircled{\small 4}} &
  \textcolor{CAmber}{\textbf{黄金选题策略}} &
  寻找「当前技术刚好够得着的前沿」——即C类「理论圣杯型」研究。65个样本中仅1例(古DNA表观遗传印记重建),极其稀有但具有最高综合价值。\\
  \rowcolor{CGreenL}
  \textbf{\textcircled{\small 5}} &
  \textcolor{CGreen}{\textbf{算法揭示真实代价}} &
  强制提升可行性/合理性(v3)使两目标均值提升 $+1.1\sim+2.7$ 分,但长尾度/前沿性各损失约0.7分。这个代价真实且不可避免——除非找到C类研究方向。\\
\end{tabularx}
\caption{核心结论与科研选题启示(共5条)}
\end{table}

\vspace{6pt}
\noindent\textcolor{CLGray}{\rule{\linewidth}{0.5pt}}
\begin{center}
  \small\color{CGray}
  数据来源:MOEA/D $\times$ Claude-sonnet-4-5 演化系统 $\cdot$
  v2(7目标,种群=20,10代)$+$ v3(强化约束,种群=20,6代)\\
  样本数:65 $\cdot$ Pearson 相关系数,双侧检验
\end{center}

\end{document}
